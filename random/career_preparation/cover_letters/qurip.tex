\documentclass[11pt]{letter} % 10pt font size default, 11pt and 12pt are also possible
\renewcommand*\familydefault{\sfdefault} 
\usepackage[margin=1in]{geometry}

% \address{610 Jackson St. \\ Santa Clara, CA 95050 \\ (408)597-4214 } % Your address and phone number
\address{6103 S University Ave. \\ Chicago, IL 60637 \\ (408)597-4214 } % Your address and phone number

\begin{document}

\begin{letter}{}

\opening{\textbf{Dear QURIP,}}

Last spring, I started a club (ucquantum.org) for students interested in quantum information at UChicago, advised by Prof. Andrew Cleland.
I felt that there was a need for an academic club specifically for quantum technologies and NISQ computing.
This fall we began hosting weekly meetings and reaching out to various departments to find interested students.
Despite initially being slow going, our bimonthly newsletter covering news and Quantum research at UChicago has about 60 subscribers and we have a group of about 10 come each week.
The most rewarding part about this experience has been getting to explain superposition and entanglement to people.
 
Once we had learned the basics, we started planning events and reaching out to professors for when we got approved.
We toured Prof. David Schuster's superconducting qubit lab and spoke to a few professors about their work.
UChicago Quantum Society was approved to be an official club at the beginning of March.
I'm particularly excited about being able to host hackathons and panels with professors, unfortunately, students are off campus next quarter.
 
I believe that participating in QURIP will give me insights into how to better teach and engage people with the field.
I'm hoping to use these experiences to be a more effective club president.
 
As a computer science major, I have been interested in the software used to control quantum devices.
I have used many of the quantum "languages" briefly (QISKit, Cirq, Quirk, etc), but after taking a course on Quantum Computation this quarter, I wanted to write my own quantum programming interface to learn how someone might make a fully featured simulator.
 
MyQuantum is a Julia-lang package that implements the common 1-bit operators (X, Y, Z, H), arbitrary controlled gate, and an arbitrary single-qubit rotation that can be composed with a control.
I also wrote functions to test if normalized, hermitian, unitary, added documentation, and unit tests.
I am looking to extend what I currently have to use as a pre-hackathon learners repo, to get people familiar with code and notation, but not be too advanced.
 
I am incredibly excited about the future of quantum computing and feel I have the necessary qualifications
to be a valuable team member in researching NISQ devices.

\begin{flushright}
    Sincerely, \\ Anand Jain \\ University of Chicago 2021 \\ anandj@uchicago.edu
\end{flushright} 

\end{letter}
 
\end{document}
