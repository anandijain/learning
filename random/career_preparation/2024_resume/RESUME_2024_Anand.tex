\documentclass[10pt]{extarticle}
\usepackage[scaled]{helvet}
\renewcommand*\familydefault{\sfdefault} %% Only if the base font of the document is to be sans serif
\usepackage[T1]{fontenc}
\marginparwidth 0pt \oddsidemargin 0pt \evensidemargin 0pt
\marginparsep0pt \topmargin 0pt \textwidth 7.5in \textheight 9in
\addtolength{\oddsidemargin}{-.5in}
\addtolength{\evensidemargin}{-.5in}	
\setlength{\parskip}{\baselineskip}
\usepackage{enumitem}
\usepackage{hyperref}

\addtolength{\topmargin}{-1.075in}
\addtolength{\textheight}{1.75in}
\parindent=0pt
\def\ccfont{\sf}
\begin{document}

\hspace{0.145in} {\huge\bf ANAND I. JAIN} \hspace{0.4cm} 610 Jackson St., Santa Clara CA 95050 \hspace{0.1cm} 408-597-4214 \hfill anandash2@gmail.com

\begin{list}{}{\setlength{\leftmargin}{1.4in}\setlength{\labelsep}{0.1cm}\setlength{\labelwidth}{1.18in}}
\item[{\bf EDUCATION}\hfill]
{\bf University of Chicago}\hfill {\bf 2017-2020}\\
Computer Science

{\bf Courses:}
•Abstract Linear Algebra (MATH 20250) •Theory of Algorithms (CMSC 27200) •Computer Systems (CMSC 15400) •Discrete Mathematics (CMSC 27100)
•Electronics (PHYS 22600) •Inventing Interactive Devices (CMSC 23220) •Mathematical Logic (CMSC27700) •Quantum Computation (MENG 23700)

\item[\bf EXPERIENCE\hfill] 

{\bf Research Programmer, JuliaHub Inc.}\hfill {\bf Apr 2021 - July 2023}
  \setlist{nolistsep}
  \begin{itemize}[noitemsep]
    \item Implemented parsers for the CellML (CellMLToolkit.jl), Systems Biology Markup Language (SBML) (SBMLToolkit.jl), and MathML (MathML.jl) specifications in the Julia programming language. 
    %\item Experienced in high performance differential equation simulation for large state systems (>1000 ODEs).
    \item Automated simulation of \textasciitilde2000 ODE models on the BioModels.org database for performance tracking and specification compliance (\url{github.com/anandijain/BioModelsAnalysis.jl})
    \item Worked with pharma clients to translate MATLAB models to Julia for improved performance.
  \end{itemize}

{\bf Large Synoptic Survey Telescope Machine Learning Intern, Fermilab} \hfill {\bf June - Aug 2019}
  \setlist{nolistsep}
  \begin{itemize}[noitemsep]
    \item Researched the applications of neural differential equations in astronomy for the LSST.
    \item Used the PLAsTiCC Astronomical Kaggle dataset to train a neural network to approximate
    the differential equation of different astronomical objects’ light curves (brightness over time)
    \item Presented poster of my work on Neural-ODEs at 2019 LSST Conference in Arizona
    \item Source code:  \url{github.com/deepskies/cosmoNODE} - Private
  \end{itemize}


\item[\bf PROJECTS\hfill]

{\bf cas3.rs - Implementing Mathematica in the Rust Language}\hfill {\bf 200 Hours Sept-Oct 2023}
\setlist{nolistsep}
\begin{itemize}[noitemsep]
	\item Wrote the core pattern matcher, parser, and evaluator of Mathematica in Rust.
	\item Cas3 can reduce S-K combinator system expressions, simulate cellular automata, and calculate symbolic derivatives.
	\item Source code: \url{github.com/anandijain/cas3.rs} Video development: \url{https://youtube.com/playlist?list=PL79kqjVnD2EOBvsTiCQqX0ZAwx9AKiA_w&si=emnOHZQEnS98zd4m}
\end{itemize}

{\bf Building a modular synthesizer from scratch}\hfill {\bf 300 Hours Dec 2023-Feb 2024}
\setlist{nolistsep}
\begin{itemize}[noitemsep]
	\item Used kicad to implement a square and sine-wave voltage controlled oscillator, dual power supply (+- 9V), a transistor amplifier, envelope generator, and voltage controlled low-pass filter.
	\item Used the toner-transfer method to fabricate double sided through-hole PCBs of above modules.
	\item Source code:  \url{github.com/anandijain/synth.kicad} Video development: \url{https://youtube.com/playlist?list=PL79kqjVnD2ENdJDDSTUD3ZMdZPhVVu8yw&si=TNAEGmvNShpb-L0L}
\end{itemize}

{\bf Free2Pee - Free bathroom locator website in Rust and Leptos}\hfill {\bf 75 Hours Feb-Sept 2023}
\setlist{nolistsep}
\begin{itemize}[noitemsep]
	\item Website hosted on GitHub Pages that uses the OpenStreetMap database to find nearby bathrooms, then uses an open-source API to determine distance to the user, returning a table of all nearby bathrooms, metadata, and distance.
	\item Compiling to WebAssembly makes the website hostable statically and serverless.
	\item Source code: \url{https://github.com/free2pee/free2pee}
	\item Website: \url{https://free2pee.github.io/free2pee/}
\end{itemize}

{\bf AILeetcode - Evaluating GPT-3.5 on every LeetCode coding question}\hfill {\bf 75 Hours Jun-Aug 2023}
\setlist{nolistsep}
\begin{itemize}[noitemsep]
	\item Created a dataset of \textasciitilde40000 OpenAI GPT completions (\textasciitilde2000 questions x 20 languages) to study the quality of code generation of LLMs across languages
	\item Submitted answers generated by GPT to LeetCode by reverse engineering LeetCode's REST API - got into top 3000 rank in LeetCode
	\item Source code: \url{https://github.com/anandijain/leetcode_evals}
\end{itemize}

{\bf Music generation with Variational-Autoencoders (VAE) in Pytorch} \hfill {\bf30 Hours Jan-Feb 2020}
\setlist{nolistsep}
\begin{itemize}[noitemsep]
	\item Trained a VAE to find a low dimensional representation of sound segments 1-3 seconds in length
	\item Source code: \url{https://github.com/anandijain/sippyart}. 
	\item Music Outputs: \url{https://anonstandardunitofmeasurement.bandcamp.com/album/vae}
\end{itemize}

\item[\bf PUBLICATIONS\hfill]

Lang, P. F., Jain, A., \& Rackauckas, C. (2024). SBMLToolkit.jl: A Julia package for importing SBML into the SciML ecosystem. \textit{Journal of Integrative Bioinformatics}. \href{https://www.degruyter.com/document/doi/10.1515/jib-2024-0003/html}{doi:10.1515/jib-2024-0003}

Rackauckas, C., Anantharaman, R., Edelman, A., Gowda, S., Gwozdz, M., Jain, A., Laughman, C., Ma, Y., Martinuzzi, F., Pal, A., Rajput, U., Saba, E., \& Shah, V. B. (2021). Composing Modeling and Simulation with Machine Learning in Julia. \textit{arXiv:2105.05946}. Retrieved from \href{https://arxiv.org/abs/2105.05946}{arXiv:2105.05946}

%Iravanian, S., Martensen, C. J., Cheli, A., Gowda, S., Jain, A., Ma, Y., \& Rackauckas, C. (2022). Symbolic-Numeric Integration of Univariate Expressions based on Sparse Regression. \textit{CoRR}, \textit{abs/2201.12468}. Retrieved from \href{https://arxiv.org/abs/2201.12468}{arXiv:2201.12468}

\item[\bf SKILLS\hfill] Rust, Python, Julia, C, Kicad, Git, Continuous Integration, High-performance modeling and simulation of differential equations, Deep learning, MATLAB, Autodesk Fusion, Ableton Live

\end{list}

\end{document}