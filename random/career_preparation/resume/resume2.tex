%%%%%%%%%%%%%%%%%%%%%%%%%%%%%%%%%%%%%%%%%%%%%%%%%%%%%%%%%%%%%%%%%%%%%%%%%%%%%%%%
% Medium Length Graduate Curriculum Vitae
% LaTeX Template
% Version 1.2 (3/28/15)
%
% This template has been downloaded from:
% http://www.LaTeXTemplates.com
%
% Original author:
% Rensselaer Polytechnic Institute 
% (http://www.rpi.edu/dept/arc/training/latex/resumes/)
%
% Modified by:
% Daniel L Marks <xleafr@gmail.com> 3/28/2015
%
% Important note:
% This template requires the res.cls file to be in the same directory as the
% .tex file. The res.cls file provides the resume style used for structuring the
% document.
%
%%%%%%%%%%%%%%%%%%%%%%%%%%%%%%%%%%%%%%%%%%%%%%%%%%%%%%%%%%%%%%%%%%%%%%%%%%%%%%%%

%-------------------------------------------------------------------------------
%	PACKAGES AND OTHER DOCUMENT CONFIGURATIONS
%-------------------------------------------------------------------------------

%%%%%%%%%%%%%%%%%%%%%%%%%%%%%%%%%%%%%%%%%%%%%%%%%%%%%%%%%%%%%%%%%%%%%%%%%%%%%%%%
% You can have multiple style options the legal options ones are:
%
%   centered:	the name and address are centered at the top of the page 
%				(default)
%
%   line:		the name is the left with a horizontal line then the address to
%				the right
%
%   overlapped:	the section titles overlap the body text (default)
%
%   margin:		the section titles are to the left of the body text
%		
%   11pt:		use 11 point fonts instead of 10 point fonts
%
%   12pt:		use 12 point fonts instead of 10 point fonts
%
%%%%%%%%%%%%%%%%%%%%%%%%%%%%%%%%%%%%%%%%%%%%%%%%%%%%%%%%%%%%%%%%%%%%%%%%%%%%%%%%

\documentclass[margin]{res}  

% Default font is the helvetica postscript font
% \usepackage{avant}
% \setupbodyfont[ss]

% Increase text height
\textheight=700pt

\begin{document}

%-------------------------------------------------------------------------------
%	NAME AND ADDRESS SECTION
%-------------------------------------------------------------------------------
\name{Anand Jain}

% Note that addresses can be used for other contact information:
% -phone numbers
% -email addresses
% -linked-in profile

\address{
github, linkedin : anandijain\\
site: anandj.net\\
}
\address{anandj@uchicago.edu \\(408)597-4214\\}

% Uncomment to add a third address
%\address{Address 3 line 1\\Address 3 line 2\\Address 3 line 3}
%-------------------------------------------------------------------------------

\begin{resume}

%-------------------------------------------------------------------------------
%	EDUCATION SECTION
%-------------------------------------------------------------------------------
\section{EDUCATION}
\textbf{University of Chicago}
{\sl B.S.}, Computer Science. \hfill \textbf{Expected Jun 2021}\\
\textbf{Santa Clara High School} \hfill \textbf{2017}

\section{COURSES}
\par
\normalfont{\textbullet{Abstract Linear Algebra}}
\normalfont{\textbullet{Algorithms}}
\normalfont{\textbullet{Computer Systems}}
\normalfont{\textbullet{Discrete Math}}\\
\normalfont{\textbullet{Electronics}}
\normalfont{\textbullet{Inventing Interactive Devices}}
\normalfont{\textbullet{Mathematical Logic}}\\
\normalfont{\textbullet{Molecular Engineering}}
\normalfont{\textbullet{Quantum Computation}}

%-------------------------------------------------------------------------------
%	COMPUTER SKILLS SECTION
%-------------------------------------------------------------------------------
\section{SKILLS}
\textbf{Languages:} Python, Julia, Go, Bash, C/C++, SQL\\
\textbf{Packages:} PyTorch, Gym, TensorFlow, Scikit-Learn, Pandas, Flask \\
\textbf{Spoken:} Fluent English. Classroom Hindi, Spanish, and Mandarin

%-------------------------------------------------------------------------------
% Modify the format of each position
\begin{format}
\title{l}\\
\dates{l}\location{r}\\
\body\\
\end{format}
%-------------------------------------------------------------------------------
\section{EXPERIENCE}
\textbf{Fermilab - LSST Machine Learning Intern\hfill Jun - Aug 2019}\\
\textbullet{Researched applications of neural differential equations in astronomy for the Large Synoptic Survey Telescope (LSST)\\}
\textbullet{Used PLAsTiCC Astronomical Kaggle dataset to train a neural network to approximate the differential equation of different objects' light curves (brightness over time)}\\
\textbullet{Presented poster of my work on Neural-ODEs at 2019 LSST Conference in Arizona}\\
\textbullet{Worked with peers and mentors to create a high level API for fast prototyping and ensemble training of neural networks for astronomy datasets, primarily in PyTorch}
\begin{itemize}
    \item \textbf{Tools:} TorchDiffEq, DifferentialEquations.jl, PyTorch, TensorFlow, Matplotlib, Astropy, Python, Julia
    \item \textbf{Link :} github.com/deepskies/cosmoNODE and /dsutils

\end{itemize}

\section{PROJECTS}

\textbf{gym-sips: machine learning in sports betting on google cloud}\\ 
\textbullet{Collected $\sim$1000 games of NFL, NHL, NBA, and MLB odds and scores on Linux VMs}\\
\textbullet{Trained/tested LSTM model to predict odds and scores on $\sim$$5\times 10^{5}$ rows of 20 features}\\
\textbullet{Created a discrete and continuous action space gym environment where agent either picks one team to place money on or allocates some amount to each}\\
\textbullet{Tested the PPO, SAC, and DDPG algorithms from OpenAI's Spinning Up in RL}\\
\textbullet{Agent learns to hedge across time and returns a positive net reward on test set}

\begin{itemize}
\item \textbf{Tools:} pytorch, gym, spinningup
\end{itemize}
\begin{itemize}
\item \textbf{Link :} github.com/anandijain/sips and /gym-sips
\end{itemize}
\textbf{sippyart: variational-autoencoders for music generation}\\ 
\textbullet{Built tool to recreate images and 1-2 second sections of audio using convolutional variational autoencoders running on GPU}\\
\textbullet{Model learns to recreate melody better than rhythm, examples in README}\\
\begin{itemize}
\item \textbf{Tools:} pytorch, torchaudio, torchvision
\end{itemize}
\begin{itemize}
\item \textbf{Link :} github.com/anandijain/sippyart
\end{itemize}

\section{ACTIVITIES}

\textbf{UCQuantum (.org) - Founder/President}\hfill \textbf{Aug 2019 - Now}\\
\textbullet{Undergraduate Student Organization of $\sim$50 facebook group members, $\sim$10 active}\\
\textbullet{Toured Prof. David Schuster's lab and learned about cooling to superconducing temperatures and software interfaces to quantum computers}\\
\textbullet{Planning a hackathon in spring to make Prof. Schuster's computers compatible with QuTiP and qiskit}
%-------------------------------------------------------------------------------

\end{resume}
\(\)\end{document}