%%%%%%%%%%%%%%%%%%%%%%%%%%%%%%%%%%%%%%%%%%%%%%%%%%%%%%%%%%%%%%%%%%%%%%%%%%%%%%%%
% Medium Length Graduate Curriculum Vitae
% LaTeX Template
% Version 1.2 (3/28/15)
%
% This template has been downloaded from:
% http://www.LaTeXTemplates.com
%
% Original author:
% Rensselaer Polytechnic Institute 
% (http://www.rpi.edu/dept/arc/training/latex/resumes/)
%
% Modified by:
% Daniel L Marks <xleafr@gmail.com> 3/28/2015
%
% Important note:
% This template requires the res.cls file to be in the same directory as the
% .tex file. The res.cls file provides the resume style used for structuring the
% document.
%
%%%%%%%%%%%%%%%%%%%%%%%%%%%%%%%%%%%%%%%%%%%%%%%%%%%%%%%%%%%%%%%%%%%%%%%%%%%%%%%%

%-------------------------------------------------------------------------------
%	PACKAGES AND OTHER DOCUMENT CONFIGURATIONS
%-------------------------------------------------------------------------------

%%%%%%%%%%%%%%%%%%%%%%%%%%%%%%%%%%%%%%%%%%%%%%%%%%%%%%%%%%%%%%%%%%%%%%%%%%%%%%%%
% You can have multiple style options the legal options ones are:
%
%   centered:	the name and address are centered at the top of the page 
%				(default)
%
%   line:		the name is the left with a horizontal line then the address to
%				the right
%
%   overlapped:	the section titles overlap the body text (default)
%
%   margin:		the section titles are to the left of the body text
%		
%   11pt:		use 11 point fonts instead of 10 point fonts
%
%   12pt:		use 12 point fonts instead of 10 point fonts
%
%%%%%%%%%%%%%%%%%%%%%%%%%%%%%%%%%%%%%%%%%%%%%%%%%%%%%%%%%%%%%%%%%%%%%%%%%%%%%%%%

\documentclass[margin]{res}  

% Default font is the helvetica postscript font
\usepackage{helvet}

% Increase text height
\textheight=700pt

\begin{document}

%-------------------------------------------------------------------------------
%	NAME AND ADDRESS SECTION
%-------------------------------------------------------------------------------
\name{Anand Jain}

% Note that addresses can be used for other contact information:
% -phone numbers
% -email addresses
% -linked-in profile

\address{
github, linkedin : anandijain\\
site: anandj.net\\
}
\address{anandj@uchicago.edu \\(408)597-4214\\}

% Uncomment to add a third address
%\address{Address 3 line 1\\Address 3 line 2\\Address 3 line 3}
%-------------------------------------------------------------------------------

\begin{resume}

%-------------------------------------------------------------------------------
%	EDUCATION SECTION
%-------------------------------------------------------------------------------
\section{EDUCATION}
\textbf{University of Chicago}
{\sl B.S.}, Computer Science. \hfill \textbf{Expected Jun, 2021}\\
\textbf{Santa Clara High School} \hfill \textbf{2017}

\section{COURSES}
\par
\normalfont{\textbullet{Algorithms}}
\normalfont{\textbullet{Discrete Math}}
\normalfont{\textbullet{Abstract Linear Algebra}}
\normalfont{\textbullet{Mathematical Logic}}
\normalfont{\textbullet{Inventing Interactive Devices}}
\normalfont{\textbullet{Electronics}}
\normalfont{\textbullet{Computer Systems}}

%-------------------------------------------------------------------------------
%	COMPUTER SKILLS SECTION
%-------------------------------------------------------------------------------
\section{SKILLS}
\textbf{Languages:} Python, Julia, Go, Bash, C\\
\textbf{Packages:} PyTorch, Gym, TensorFlow, Scikit-Learn, Pandas, Flask \\
\textbf{Spoken:} Fluent English. Classroom Hindi and Spanish

%-------------------------------------------------------------------------------
% Modify the format of each position
\begin{format}
\title{l}\\
\dates{l}\location{r}\\
\body\\
\end{format}
%-------------------------------------------------------------------------------
\section{EXPERIENCE}
\textbf{Fermilab - LSST Machine Learning Intern\hfill Jun 24 - Aug 31 2019}\\
\textbullet{Researched the applications of neural differential equations to astronomy for the Large Synoptic Survey Telescope\\}
\textbullet{Used the PLAsTiCC Astronomical Kaggle dataset to train a neural network to approximate the differential equation of different astronomical objects' light curves (brightness over time)}\\
\textbullet{Presented a poster of my work on Neural-ODEs at the 2019 LSST Conference in Arizona}\\
\textbullet{Worked with peers and mentors to create a high level API for fast prototyping and ensemble training of neural networks for astronomy datasets, primarily in PyTorch}
\begin{itemize}
    \item \textbf{Tools:} TorchDiffEq, DifferentialEquations.jl, PyTorch, TensorFlow, Matplotlib, Astropy, Python, Julia
\end{itemize}
\section{PROJECTS}
\textbf{Sips: Sports Data Tool on Google Cloud\hfill Oct 2018 - Now}\\
\textbullet{Created Python package to track odds and statistics for a variety of sports\\}
\textbullet{Have collected over 1 GBs of data using Linux virtual machines on Google Cloud }\\
\textbullet{Trained TensorFlow win/loss classification and LSTM odds prediction neural network models on Sips data}\\
\textbullet{Wrote Flask app to visualize odds using Matplotlib}
\begin{itemize}
\item \textbf{Tools:} Python, Requests, Beautiful Soup, TensorFlow, Flask, Matplotlib
\end{itemize}
\begin{itemize}
\item \textbf{Link :} github.com/anandijain/sips
\end{itemize}
\textbf{Gym-Sips: Reinforcement Learning Environment\hfill Feb 2019 - Now}\\
\textbullet{Custom reinforcement learning environments using the OpenAI Gym package for the output data of sips}\\
\textbullet{Agent takes actions to buy the home team or the away team moneyline, and receives reward at the end of the game}\\
\textbullet{Agent learns hedging strategies to mitigate risk}

\begin{itemize}
\item \textbf{Tools:} Python, Gym, TensorFlow, TF-Agents
\end{itemize}
\begin{itemize}
\item \textbf{Link :} github.com/anandijain/gym-sips
\end{itemize}

\section{ACTIVITIES}

\textbf{UCQuantum (.org) - Founder/President}
\hfill \textbf{Aug 2019 - Now}\\
\textbullet{Unofficial club applying to become an official RSO for UChicago undergrads interested in quantum computing}\\
\textbullet{We have 50 facebook group members and 10 active members}\\
\textbullet{We are planning to host talks, hackathons, and lab tours with faculty on campus}\\
%-------------------------------------------------------------------------------

\end{resume}
\(\)\end{document}