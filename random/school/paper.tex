\documentclass[12pt]{article}
\usepackage[headheight=12pt,headsep=0pt]{geometry}
\renewcommand*\familydefault{\sfdefault}
\usepackage{hyperref}
\usepackage{fullpage}
\usepackage{amsfonts}
\usepackage{amsmath}
\usepackage{amssymb}
\usepackage{fancyhdr}
\usepackage{lipsum} % only for showing some sample text
\setlength\parindent{24pt}
\linespread{2}
\fancyhf{} % clear all header and footers
\renewcommand{\headrulewidth}{0pt} % remove the header rule
\rhead{\thepage}
\pagestyle{fancy}
% \author{Anand Jain}
% \date{3 27 20}

% \title{\small Sosc Winter Quarter \\ 
\usepackage{lmodern}

\begin{document}
\noindent Anand Jain \\ 
Ali Feser \\
3/27/20 \\
Power, Identity, and Resistance-2 
\begin{center}
    MCM' and the Tendency of the Rate of Profit
\end{center}
\begin{itemize} 
    \item In Capital, Marx asks how a capitalist can purchase something with a certain value and sell it at a higher value. 
    \item Marx concludes that the defining characteristic of a capitalistic society is the ruling class' ability to valorize their capital, defined by the equation MCM'. 
    \item Captialist ownership of surplus value, the increasing proportion of constant capital to variable capital, and the declining rate of profit provide as the basis for Marx's claim that the capitalist mode of production will necessarily be replaced.

    \item I will define the model that Smith presents, contrast with Marx, and show that the structural contradictions in capitalism that Marx presents as necessitating a post-capitalist economy follow from the calculation of $\lim_{t \to \infty}$, but fails to provide bounds on the timing of this eventuality. 
    \item Instead of viewing the declining rate of profit as a indicator of capitalisms inevitable demise, I demonstrate the tendency of the rate of profit to fall is the enabling factor that allows for surplus value to be redistributed, in the form of lower cost technology, enabling wage-laborers to enter the hussle.
    \item A key feature of Smith's model claims maximal individual liberty is ideal for economic and technological growth because the function optimal\_action(someone\_else) is incomputable. 
    \item Smith defines growth based off of the total value of commodities, claiming that services rarely have value after the service is complete. \textbf{quote}
    \item Smith describes a commodity by it's exchange value/worth, $W$, derived by the sum of the constant capital, variable capital, and surplus value, given by $c + L + s = W$. \textbf{quote}
    \item Ignoring economies of scale for this paper, I assume linearity with commodity quantity $q$ to arrive at $\forall q \in \mathbb{N}, \forall{cLs} \in \mathbb{R^{+}}, q(c + L + s) = qW = qc + qL + qs $. \textbf{quote}
    \item The nation's economy will be defined as a graph on a function $f(t) = G_t(V_t, E_t)$. 
    \item Defining this graph on a function allows for any vertex in set of vertices $V$, and for any edge in the set of edges $E$ to change continuously.
    \item Now to define what each vertex and edge is (person and instantaneous time trade respectively): We will assume, following Smith's claim that there are uncountably many complexities regarding human action that necesitate individual liberty. \textbf{quote} 
    \item We can define three important types (commodities, people, and trades) that generate this model. 
    \item A Commodity is a tuple containing: quantity/amount $q$, per-unit constant capital cost $c$, the per-unit variable capital cost $L$, and resulting per-unit surplus value $s$. 
	\item Let the set of all commodities (past and present) be an infinite ordered-set denoted $\boldsymbol{C}$ with cardinality $m = |\boldsymbol{C}|$. \textbf{quote}.
 
    \item A Person $i$ at a given time $t$ is denoted $P_{it}$ defined by: a list of values for all commodities (called a Portfolio) $A_{it} = ((q, c, L, s)_{1}, (q, c, L, s)_{2},...,(q, c, L, s)_{m})$, an arbitrary uncountable bitstring (0s and 1s) $B_{it}$ that cannot be computed a priori by someone else and is assumed to be random, and a function, where given one's own bitstring, and a proposed trade $\hat{e_{ijt}}$, a binary decision is computed instantaneously, $g_{it} : (B_{it}, \hat{e_{ijt}}) \mapsto \{x | x \in  \{0, 1\}\}$. 
	\item (does this also mean that from the set of all decisions forall possible trades exists a bijection to the persons bitstring) probably? 
    \item A Trade at time $t$ is an instantaneous bidirected edge between the two Persons and N-Person trades are reduced to atomic instantaneous 2 Person Trades and no overdraws in Trades are allowed. 
    \item Exercising a use-value can be represented as a vertex with an edge to itself, called a loop, in which a trade with one's own commodites consumes their existence.
    \item Notationally: $q_{ijt_{a}} = V_{it_{a}}.A_{jq} =$ the quantity $q$ of commodity $j$ that person $i$ owns at time $a$, similarly for the constant capital, variable capital, and surplus value.
    \item An example of labor in this model: $P_i$ Trades with $P_j$ for assets $A$ and Trades, typically money, with $P_k$.
    \item $P_k$ expends their own assets through work/labour by a series of Trades with $P_i$ in which $A$ is augmented into a new commodity with surplus value, owned by $P_i$.  
    \item Note, Money::Commodity, but acts as a commodity that serves as a primary mode of conversion for commodities associated with a consumable use-value, as the use-value of money is to employ it and generate a surplus value, given by MCM'.
    \item \textbf{quote on monetary value}
    \item This quote is a based off of the mean-value theorem (MVT); we can convert all commodities into a single value for monetary wealth by summing up $q(c+L+s)$ for all commodities, for a given set of people, trades, etc.
    \item If we let $\boldsymbol{V} :=\bigcup\limits_{t \leq t_a} V_t \in G_t$ and $\boldsymbol{E} :=\bigcup\limits_{t \leq t_a} E_t \in G_t$ maximal set of all people and trades, then we can define the wealth of a nation at time $t_a$ as $\sum\limits_i^n$ $\sum\limits_j^m$ $q_{ijt_{a}} * (c_{ijt_{a}} + L_{ijt_{a}} + s_{ijt_{a}})$. 
    \item The reason that it is not a sum over all $t < t_a$ is that the surplus value accumulated through production is stored in the costant capital of the current commodities owned by people. \textbf{quote}
    \item This framework makes easy to interpret the valorization of capital.
    \item For example, if a Person $i$ at time $t$ with commodities/assets $A$ can employ their capital in a succession of Trades and conclude with more Commodities than starting, then they have valorized their capital via arbitrage.
    \item  Profitting off of the discrepancies in prices is not a zero sum game, as the arbitrageur is forcing price stability by taking profits from commodity producers, who subsequently need to employ more constant capital to reduce the per-unit variable capital cost.
    \item We now want to extend this model to account for the class dynamics of MCM' that Marx presents as necessitating a post-capitalist mode of production and show that the discretization of individuals into homogeneous classes violates Smith's key assertion that it is impossible to calculate the optimal action for any given individual. 
    \item \textbf{quote on MCM' and classes}
    \item \textbf{something about supply demand and normal distributions}
    \item Marx claims that the worker is completely reduced to selling their labor for wages that are spent entirely on the means of subsistence and reproduction. \textbf{quote}
    \item Let $\alpha$ be a set of assets defining the means of subsistence, and therefore a death condition given by: $\{A_i - x < 0| x \in \alpha \}$  
    \item We want to extend this model to explain labor/class relations, the labor theory of value, and the relative dimunition of the variable part of capital to constant capital.
    \item We can extend the type Person, with an attribute characterizing their class, where a Person is a wage-laborer $\beta$ or a capitalist $\gamma$, $P_i::\beta$ and $P_i::\gamma$ respectively.
    \item \textbf{Quote on use value being equivalent to exchange value}
    \item Marx identifies the apparent nonlinear behavior of the proportion of constant capital to variable capital as technology advances.
    \item  \textbf{quote}
    \item However, it is not clear that this eventuality is one that will necessarily be witnessed.
    \item Zeno's paradox is a simple analogy for the amount of time it will take for the proportion of constant capital to entirely encompass that of variable capital, which is the real precondition for a post-capitalist mode of production.
	\item We can only say what the behavior is of $\lim_{t \to \infty}$, but not when it will happen.
\end{itemize}

% thoughts/misc
\begin{itemize}
	\item It seems like a big problem is addressing market efficiency, because largely as old technologies/commodities with depreciated value, may be more valueable as different technology comes about. Difference in market price and value
	\item In order to support a multi-currency model, it makes no difference that exchanging money is represented by a Trade, since any money can be expressed as the commodities of equivalent exchange value (really market price at current time).
	\item In Marx's model, we see that the distinction in constant and variable capital can be represented as the component parts of the total capital employed by a Person on the means of production and labor respectively. To fit the model to this, we can express capital-labor exchanges (commodity production) #####? 
	\item It arises from a violation of Cauchy-Schwarz. Marx assumes that it is never the case that the labor of the worker is for goods that he actually consumes for subsistence. Dually, this violates that the worker would ever overwork themselves out of joy of the labor. If they are free laboreres, however, exploited to the limits of subsistence, can in improve the production process.
	\item How do we explain hourly wage, contracts, etc as commodity exchange. A contract's value is largely in the character of the judicial system.
	\item Need something on cooperation
	\item money
	\item to say that a capitalist never becomes a wage laborer or vice versa is wrong. marx freq writes that the capitalists aim is capital augmentation, not necesarily commodity production. however, where does he address that herein lies the labor of capitalists-asset allocation.
	\item the inner product of two uncountable vectors is still \in \mathbb{R}. but the degree of error for approximation this use-value is the challenge.
	\item it is clear that the distribution of portfolios will largely consist of subsistence commodities to be used for use-value, and that this sparsity prevents the exchange of commodities between traders when neither want the other's commodity for it's use value. this brings about the money form of commodities. it is this shift that makes distinct a capitalist mode of production.
	\item you can expend labor and capital to gain information about the characteristics of a commodity that give a better approximation of its value, and subsequently market price. 
	\item to assume that a unified/collective revolution will take place is logistically impossible. it is much more likely that given the inherent flaws in commodity prices, the worst 
	\item   $can_create_ai()$?  $revolution=true$ : $revolution=false$ this is because the worker would have no barganing power, and theres no job that an ai couldn't do.
	\item alienation can be described by labor Trades in this model. where alienation is a vertex with a loop and an edge to another vertex. as this demonstrates the resulting commodity not being owned by the worker.
	\item @430 selling labor power is a basis but not the only.  
	\item @430 2 if self-expansion of capital $~$ to $#$ of ppl displaced by it. does this imply that the expansion of capital is $~$ expansion of human knowledge?  
	\item the soln is knxwledge itself
	\item if the worker finds themselves dissatisfied with their life, but willing to give up a portion of their wages to remedy the problem, then there exists a market to deliver this want. 
	thots
	\item social commodities (experiences/mutual culture) are in a similar position to currency, in that they don't exhibit any use-value and are akin to a service to Smith. This has the analogy of prices stabilizing as more of a given commodity is translated into various other commodities, likewise the exchange of culture is exhibited by human connectivity. 
\end{itemize}


% quotes 
\begin{enumerate}
    \item Smith 20: Every man thus lives by exchanging, or becomes in some measure a merchant, and the society itself grows to be what is properly a commercial society,
    \item Marx 35: an immense accumulation of commodities,* its unit being a single commodity. (demonstrates $\in \mathbb{N}$, capitalist basis of value is commodities)
    \item Marx 35: A commodity is, in the first place, an object outside us, a thing that by its properties satisfies human wants of some sort or another.
	\item 35: To discover the various uses of things is the work of history.
    \item Marx 36: When treating of use-value, we always assume to be dealing with definite quantities, such as dozens of watches, yards of linen, or tons of iron.
	\item 38: but as exchange-values they are merely different quantities, and
		consequently do not contain an atom of use-value. note - quantities of exchange is separate
    \item Marx 39: The labour, however, that forms the substance of value,
is homogeneous human labour, expenditure of one uniform la-
bour-power. The total labour-power of society, which is embodied
in the sum total of the values of all commodities produced by that
society, counts here as one homogeneous mass of human labour-
power, composed though it be of innumerable individual units.
Each of these units is the same as any other, so far as it has the
character of the average labour-power of society, and takes effect as
such; that is, so far as it requires for producing a commodity, no
more time than is needed on an average, no more than is socially
necessary. The labour-time socially necessary is that required to
produce an article under the normal conditions of production, and
with the average degree of skill and intensity prevalent at the time.
	\item 40: The value of a commodity, there-
fore, varies directly as the quantity, and inversely as the produc
tiveness, of the labour incorporated in it. 
	\item 40: A thing can be a use-value, without having value. This is
the case whenever its utility to man is not due to labour. Such
are air, virgin soil, uatural meadows, c. A thing can be useful,
and the product of human labour, without being a commodity.
Whoever directly satisfies his wants with the produce of his own
labour, creates, indeed, use-values, but not commodities. 
In order to produce the latter, he must not only produce use-values, but
use-values for others, social use-values. 
    \item Marx 40: The value of a commodity would therefore remain constant,
if the labour-time required for its production also remained
constant. But the latter changes with every variation in the
productiveness of labour. 
    \item Marx 46: The same change in productive power, which
	increases the fruitfulness of labour, and, in consequence, the
	quantity of use-values produced by that labour, will diminish
	the total value of this increased quantity of use-values, provided
	such change shorten the total labour-time necessary for their
	production; and vice versa.
	\item 88: In the course of time, therefore, some
portion at least of the products of labour must be produced with
a special view to exchange. From that moment the distinction
becomes firmly established between the utility of an object for
the purposes of consumption, and its utility for the purposes of ex-
change. Its use-value becomes distinguished from its exchange-
value. 
	\item 88: In the direct barter of products, each commodity is directly
a means of exchange to its owner, and to all other persons an
equivalent, but that only in so far as it has use-value for them.
At this stage, therefore, the articles exchanged do not acquire
a value-form independent of their own use-value, or of the individ-
ual needs of the exchangers. The necessity for a value-form grows
with the increasing number and variety of the commodities ex-
changed. The problem and the means of solution arise simultane-
ously. 
	\item 89: Since all commodities are merely particular equivalents of
money, the latter being their universal equivalent, they, with
regard to the latter as the universal commodity, play the parts
of particular commodities. 
    \item Marx 210: The value of a commodity, it is true, is determined by the quan-
tity of labour contained in it, but this quantity is itself limited
by social conditions. If the time socially necessary for the pro-
duction of any commodity alters— and a given weight of cotton
represents, after a bad harvest, more labour than after a good
one — all previously existing commodities of the same class are
affected, because they are, as it were, only individuals of the
species, * and their value at any given time is measured by the
labour socially necessary, i.e., by the labour necessary for their
production under the then existing social conditions.
    \item Marx 216: Now since his
work forms part of a system, based on the social division of
labour, he does not directly produce the actual necessaries which
he himself consumes; he produces instead a particular commod-
ity, yarn for example, whose value is equal to the value of those
necessaries or of the money with which they can be bought. The
portion of his day's labour devoted to this purpose, will be
greater or less, in proportion to the value of the necessaries that he
daily requires on an average, or, what amounts to the same thing,
in proportion to the labour-time required on an average to produce
them.
    \item Marx 218: the values of the variable capital
and of the labour-power purchased by that capital are equal,
and the value of this labour-power determines the necessary
portion of the working-day; and since, on the other hand, the
surplus-value is determined by the surplus portion of the work-
ing-day, it follows that surplus-value bears the same ratio to
variable capital, that surplus-labour does to necessary labour.
    \item Marx 244: in any given economic formation of so-
ciety, where not the exchange-value but the use-value of the prod-
uct predominates, surplus-labour will be limited by a given set
of wants which may be greater or less
    \item 321:  The object of all development of the pro-
ductiveness of labour, within the limits of capitalist production,
is to shorten that part of the working-day, during which the work-
man must labour for his own benefit, and by that very shortening,
to lengthen the other part of the day, during which he is at liberty
to work gratis for the capitalist. 
	\item 323 The labour realised in value, is labour of an average social quality;
		is consequently the expenditure of average labour-power.
	\item 376 The union of all these simple instruments, set in motion by a single
motor, constitutes a machine. Babbage
	\item 406 Machinery produces relative surplus-value; not only by di-
rectly depreciating the value of labour-power, and by indirectly
cheapening the same through cheapening the commodities that
enter into its reproduction, but also, when it is first introduced
sporadically into an industry, by converting the labour employed
by the owner of that machinery, into labour of a higher degree and
greater efficacy, by raising the social value of the article produced
above its individual value, and thus enabling the capitalist to
replace the value of a day's labour-power by a smaller portion of
the value of a day's product.
	\item 429 It took both time and experience be-
fore the workpeople learnt to distinguish between machinery and
its employment by capital, and to direct their attacks, not against
the material instruments of production, but against the mode
in which they are used.
	\item 430 The whole system of capitalist production is based
on the fact that the workman sells his labour-power as a commodity.
	\item 430 The self-expansion of capital by means of machinery is
thenceforward directly proportional to the number of the work-
people, whose means of livelihood have been destroyed by that
machinery. 
	
	\item 440 
	\item 461 analysing the process of production into its constituent phases,
and of solving the problems thus proposed by the application of
mechanics, of chemistry, and of the whole range of the natural
sciences, becomes the determining principle everywhere. 
	\item 530 Only by suppressing the
capitalist form of production could the length of the working-day
be reduced to the necessary labour-time. But, even in that case,
the latter would extend its limits. On the one hand, because the
notion of "means of subsistence" would considerably expand, and
the labourer would lay claim to an altogether different standard of
life. On the other hand, because a part of what is now surplus-
labour, would then count as necessary labour; I mean the labour
of forming a fund for reserve and accumulation.
	\item 530 2 The capitalist mode of production, while on the
one hand, enforcing economy in each individual business, on the oth-
er hand, begets, by its anarchical system of competition, the most
outrageous squandering of labour-power and of the social means of
production, not to mention the creation of a vast number of employ-
ments, at present indispensable, but in themselves superfluous.
	\item 530 3 The intensity and productiveness of labour being given, the
time which society is bound to devote to material production is shorter, and as a consequence, the time at its disposal for the
free development, intellectual and social, of the individual is
greater, in proportion as the work is more and more evenly divid-
ed among all the able-bodied members of society, and as a par-
ticular class is more and more deprived of the power to shift the
natural burden of labour from its own shoulders to those of an-
other layer of society. In this direction, the shortening of the work-
ing-day finds at last a limit in the generalisation of labour. In
capitalist society spare time is acquired for one class by converting
the whole life-time of the masses into labour-time.
 
	\item  
	\item 554 society can no more cease to produce
than it can cease to consume. When viewed, therefore, as a con-
nected whole, and as flowing on with incessant renewal, every
social process of production is, at the same time, a process of re-
production.
	\item 569: Variable capital is therefore only a particular historical form
		of appearance of the fund for providing the necessaries of life, or
		the labour-fund which the labourer requires for the maintenance of
		himself and family, and which, whatever be the system of social
		production, he must himself produce and reproduce.
	\item Marx 612: The composition of capital is to be understood in a twofold sense. On the side of value, it is determined by the proportion in which it is divided into constant capital or value of the means of production, and variable capital or value of labour-power, the sum total of wages. On the side of material, as it functions in the process of production, ll capital is divided into means of production and living labour-power
    \item Marx 622: the degree of productivity of labour, in a given society, is expressed in the relative extent of the means of production that one labourer, during a given time 
    \item Marx 622: the growing extent of the means of production, as compared with the labour-power in corporated with them, is an expression of the growing produc tiveness of labour.
    \item Marx 622: This change in the technical composition of capital, this
growth in the mass of means of production, as compared with
the mass of the labour-power that vivifies them, is reflected again
in its value-composition, by the increase of the constant consti-
tuent of capital at the expense of its variable constituent. 
    \item Marx 624: Every individual capital is a larger or smaller concentration of means of production, with a corresponding command over a larger or smaller labour-army. 
	\item 627: In any given branch of industry cen-
tralisation would reach its extreme limit if all the individual
capitals invested in it were fused into a single capital. 1 In a
given society the limit would be reached only when the entire
social capital was united in the hands of either a single capitalist
or a single capitalist company.


    \item 
\end{enumerate}


\begin{itemize}
    \item \href{https://en.wikipedia.org/wiki/Graph_(abstract_data_type)}{Graphs}
    \item \href{https://archive.org/details/HannahArendt/page/n1/mode/2up}
\end{itemize}

% and are consequent from discretizing the populous into homogeneous classes and that collective ownership

% let $n := |\boldsymbol{V}|$ being the total number of people to have existed in this society.
% $\forall v, e \in (\boldsymbol{V}, \boldsymbol{E}), v::$Person$, e::$Trade, where Person at time T $:= (A_{t}, B_{t}, u_{t}, e_{t})$ where $u$ is a function that takes in 
\end{document}
