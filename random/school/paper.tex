\documentclass[12pt]{article}
\usepackage[headheight=12pt,headsep=0pt]{geometry}
\renewcommand*\familydefault{\sfdefault}
\usepackage{hyperref}
\usepackage{fullpage}
\usepackage{amsfonts}
\usepackage{amsmath}
\usepackage{amssymb}
\usepackage{fancyhdr}
\usepackage{lipsum} % only for showing some sample text
\setlength\parindent{24pt}
\linespread{2}
\fancyhf{} % clear all header and footers
\renewcommand{\headrulewidth}{0pt} % remove the header rule
\rhead{\thepage}
\pagestyle{fancy}
% \author{Anand Jain}
% \date{3 27 20}

% \title{\small Sosc Winter Quarter \\ 
\usepackage{lmodern}

\begin{document}
\noindent Anand Jain \\ 
Ali Feser \\
3/27/20 \\
Power, Identity, and Resistance-2 
\begin{center}
    MCM' and the Tendency of the Rate of Profit
\end{center}
\begin{itemize} 
    \item In Capital, Marx asks how a capitalist can purchase something with a certain value and sell it at a higher value. 
    \item Marx concludes that the defining characteristic of a capitalistic society is the ruling class' ability to valorize their capital, defined by the equation MCM'. 
    \item Captialist ownership of surplus value, the increasing proportion of constant capital to variable capital, and the declining rate of profit provide as the basis for Marx's claim that the capitalist mode of production will necessarily be replaced.

    \item I will define the model that Smith presents, contrast with Marx, and show that the structural contradictions in capitalism that Marx presents as necessitating a post-capitalist economy follow from the calculation of $\lim_{t \to \infty}$, but fails to provide bounds on the timing of this eventuality. 
    \item Instead of viewing the declining rate of profit as a indicator of capitalisms inevitable demise, I demonstrate the tendency of the rate of profit to fall is the enabling factor that allows for surplus value to be redistributed, in the form of lower cost technology, enabling wage-laborers to enter the hussle.
    \item A key feature of Smith's model claims maximal individual liberty is ideal for economic and technological growth because the function optimal\_action(someone\_else) is incomputable. 
    \item Smith defines growth based off of the total value of commodities, claiming that services rarely have value after the service is complete. \textbf{quote}
    \item Smith describes a commodity by it's exchange value/worth, $W$, derived by the sum of the constant capital, variable capital, and surplus value, given by $c + L + s = W$. \textbf{quote}
    \item Ignoring economies of scale for this paper, I assume linearity with commodity quantity $q$ to arrive at $\forall q \in \mathbb{N}, \forall{cLs} \in \mathbb{R^{+}}, q(c + L + s) = qW = qc + qL + qs $. \textbf{quote}
    \item The nation's economy will be defined as a graph on a function $f(t) = G_t(V_t, E_t)$. 
    \item Defining this graph on a function allows for any vertex in set of vertices $V$, and for any edge in the set of edges $E$ to change continuously.
    \item Now to define what each vertex and edge is (person and instantaneous time trade respectively): We will assume, following Smith's claim that there are uncountably many complexities regarding human action that necesitate individual liberty. \textbf{quote} 
    \item We can define three important types (commodities, people, and trades) that generate this model. 
    \item A Commodity is a tuple containing: quantity/amount $q$, per-unit constant capital cost $c$, the per-unit variable capital cost $L$, and resulting per-unit surplus value $s$. 
    \item Let the set of all commodities be an infinite ordered-set denoted $\boldsymbol{C}$ with cardinality $m = |\boldsymbol{C}|$. \textbf{quote}.
    
    \item A Person $i$ at a given time $t$ is denoted $P_{it}$ defined by: a list of values for all commodities (called assets) $A_{it} = ((q, c, L, s)_{1}, (q, c, L, s)_{2},...,(q, c, L, s)_{m})$, an arbitrary uncountable bitstring (0s and 1s) $B_{it}$ that cannot be computed a priori by someone else and is assumed to be random, and a function, where given one's own bitstring, and a proposed trade $\hat{e_{ijt}}$, a binary decision is computed instantaneously, $g_{it} : (B_{it}, \hat{e_{ijt}}) \mapsto \{x | x \in  \{0, 1\}\}$.
    \item A Trade at time $t$ is an instantaneous bidirected edge between the two Persons and N-Person trades are reduced to atomic instantaneous 2 Person Trades and no overdraws in Trades are allowed. 
    \item Exercising a use-value can be represented as a vertex with an edge to itself, called a loop, in which a trade with one's own commodites consumes their existence.
    \item Notationally: $q_{ijt_{a}} = V_{it_{a}}.A_{jq} =$ the quantity $q$ of commodity $j$ that person $i$ owns at time $a$, similarly for the constant capital, variable capital, and surplus value.
    \item An example of labor in this model: $P_i$ Trades with $P_j$ for assets $A$ and Trades, typically money, with $P_k$.
    \item $P_k$ expends their own assets through work/labour by a series of Trades with $P_i$ in which $A$ is augmented into a new commodity with surplus value, owned by $P_i$.  
    \item Note, Money::Commodity, but acts as a commodity that serves as a primary mode of conversion for commodities associated with a consumable use-value, as the use-value of money is to employ it and generate a surplus value, given by MCM'.
    \item \textbf{quote on monetary value}
    \item This quote is a based off of the mean-value theorem (MVT); we can convert all commodities into a single value for monetary wealth by summing up $q(c+L+s)$ for all commodities, for a given set of people, trades, etc.
    \item If we let $\boldsymbol{V} :=\bigcup\limits_{t \leq t_a} V_t \in G_t$ and $\boldsymbol{E} :=\bigcup\limits_{t \leq t_a} E_t \in G_t$ maximal set of all people and trades, then we can define the wealth of a nation at time $t_a$ as $\sum\limits_i^n$ $\sum\limits_j^m$ $q_{ijt_{a}} * (c_{ijt_{a}} + L_{ijt_{a}} + s_{ijt_{a}})$. 
    \item The reason that it is not a sum over all $t < t_a$ is that the surplus value accumulated through production is stored in the costant capital of the current commodities owned by people. \textbf{quote}
    \item This framework makes easy to interpret the valorization of capital.
    \item For example, if a Person $i$ at time $t$ with commodities/assets $A$ can employ their capital in a succession of Trades and conclude with more Commodities than starting, then they have valorized their capital via arbitrage.
    \item  Profitting off of the discrepancies in prices is not a zero sum game, as the arbitrageur is forcing price stability by taking profits from commodity producers, who subsequently need to employ more constant capital to reduce the per-unit variable capital cost.
    \item We now want to extend this model to account for the class dynamics of MCM' that Marx presents as necessitating a post-capitalist mode of production and show that the discretization of individuals into homogeneous classes violates Smith's key assertion that it is impossible to calculate the optimal action for any given individual. 
    \item \textbf{quote on MCM' and classes}
    \item \textbf{something about supply demand and normal distributions}
    \item Marx claims that the worker is completely reduced to selling their labor for wages that are spent entirely on the means of subsistence and reproduction. \textbf{quote}
    \item Let $\alpha$ be a set of assets defining the means of subsistence, and therefore a death condition given by: $\{A_i - x < 0| x \in \alpha \}$  
    \item We want to extend this model to explain labor/class relations, the labor theory of value, and the relative dimunition of the variable part of capital to constant capital.
    \item We can extend the type Person, with an attribute characterizing their class, where a Person is a wage-laborer $\beta$ or a capitalist $\gamma$, $P_i::\beta$ and $P_i::\gamma$ respectively.
    \item \textbf{Quote on use value being equivalent to exchange value}
    \item Marx identifies the apparent nonlinear behavior of the proportion of constant capital to variable capital as technology advances.
    \item  \textbf{quote}
    \item However, it is not clear that this eventuality is one that will necessarily be witnessed.
    \item Zeno's paradox is a simple analogy for the amount of time it will take for the proportion of constant capital to entirely encompass that of variable capital, which is the real precondition for a post-capitalist mode of production.
    \item We can only say what the behavior is of $\lim_{t \to \infty}$, but not when it will happen.

    
\end{itemize}

Marx, in assuming that 








\begin{enumerate}
    \item Smith 20: "Every man thus lives by exchanging, or becomes in some measure a merchant, and the society itself grows to be what is properly a commercial society,"
    \item Marx 41: "an immense accumulation of commodities,"* its unit being a single commodity. (demonstrates $\in \mathbb{N}$, capitalist basis of value is commodities)
    \item Marx 42: When treating of use-value, we always assume to be dealing with definite quantities, such as dozens of watches, yards of linen, or tons of iron.
\end{enumerate}


\begin{itemize}
    \item \href{https://en.wikipedia.org/wiki/Graph_(abstract_data_type)}{Graphs}

\end{itemize}

% and are consequent from discretizing the populous into homogeneous classes and that collective ownership

% let $n := |\boldsymbol{V}|$ being the total number of people to have existed in this society.
% $\forall v, e \in (\boldsymbol{V}, \boldsymbol{E}), v::$Person$, e::$Trade, where Person at time T $:= (A_{t}, B_{t}, u_{t}, e_{t})$ where $u$ is a function that takes in 
\end{document}
